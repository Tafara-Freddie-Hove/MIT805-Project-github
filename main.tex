\documentclass[conference]{IEEEtran}
\usepackage[utf8]{inputenc}
\usepackage[english]{babel}

\title{COVID-19 Public Media Dataset}
\author{Tafara Freddie Hove(u18278150)}
\date{August 2020}

\begin{document}

\maketitle
\begin{abstract}
  The COVID-19 (coronavirus) disease has affected many lives and livelihoods globally. It has impacted the socio-economic and psychological welfare of humanity.  The outbreak of the novel diseases also amplified information sharing on social networking platforms such as Twitter, Facebook and online articles. Public opinions and perceptions related to the effects of COVID-19 on health, economy and  business are being shared. The main object of this paper is to explore the COVID-19 Public Media Dataset, which is a collection of online media articles on the novel coronavirus. The dataset will be described with reference to the 4V's characteristics of big data; variety, veracity, volume, and velocity ,and the collection and processing categories.The relationships and correlation that might be mined from the dataset will be shared.
\end{abstract}

\section{Introduction}
In response to the coronavirus crisis, Anacode research company prepared a COVID-19 Public Media Dataset.  The dataset is a critical and valuable resource which can be used by researchers to generate meaningful insights and knowledge discovery to manage the coronavirus crisis. Both in academia and global science community are making use Natural Language Processing (NLP) and data mining techniques to analyse the overall impact, challenges and opportunities of the COVID-19 crisis. Hence Anacoda made this COVID-19 dataset public for researchers to exploit the data and contribute in the fight against COVID-19. 

However the exponential increase in COVID-19 literature makes it difficult for 
 researchers to retrieve quality information from very large and complex datasets. Online articles are published in large quantities day by day, which increases the size of datasets exponentially contributing to processing and data analysis a problem. \cite{Patel et al} pointed out that processing and analysing huge and complex data, or extracting valuable and quality information from large datasets is a challenging task. Hence the users of large and heterogeneous datasets must leverage AL-based techniques to extract meaningful information from such data archives \cite{Wang et al}. Hence the use of advanced technological tools such as Hadoop MapReduce will extract the hidden insights from the dataset. 

\section{Dataset}
The COVID-19 Public Media Dataset was prepared through text mining and information retrieval of COVID-19 related online text articles. Approximately, more than 20 major English language domain websites were crawled for the timespan January to April 2020. The dataset is a resource of more than 200 000 media articles which explore non-medical impacts of COVID-19 to people.  This constitutes 2G of data.   It is a combination of four datasets with comma separated values. The size and structure of the datasets makes it a Big Data. Hence, the conventional database systems cannot store and  process large datasets in acceptable time bounds \cite{Patel et al}. 


The selected webs and filtered web domains included information related to COVID-19. The articles revealed the effects of pandemic in the areas of technology, finance, business and the general impact it has cause to humanity except the medical impact. Hence the data folder contains for csv files. Each files has information for particular category of articles. 

The aim of this research is to make use of the dataset to assess and analyse the how the crisis affected lives and livelihoods socially, economically, and psychologically.In addition, we need to find out if the crisis also brought in any opportunities to business, academia and research.  For instance, has the COVID-19 data brought in any new insights such as understanding, intelligence, knowledge, perspectives and ‘actionability’ on how to manage and crate opportunities out of the crisis? Conversely, some of the expected outcomes from the analyzed data include economic decline, high levels of employment and sharp increase in fake news.  

\section{Big Data}
Big data refers to large and complex datasets of enormous size which cannot   be stored and processed by conventional resources\cite{Grolinger}. The data can be structured, semi-structured or unstructured. \cite{Owais} argued that the volume, velocity and variety of data is too big making it difficult to store, capture, manage and process. The author further alluded that big data can be found anywhere, anytime and in anyplace which makes it hard to manage and analyse using traditional applications. Data can be collected from sensors, machines, humans and business processes. Similarly the COVID-19 data is being produced globally through social media and online articles in and continues to grow unabated.

\subsection{Characteristics and Categories of Big Data}
The characteristics of big data are defined by many Vs, but in this paper we discuss 4Vs which are volume, velocity, variety and veracity. According to \cite{Owais} there are five categories of big data which are extracted from the 9Vs. The data categories are collecting data, processing data, integrity data, visualization data and worth of data.  In this paper were focus on collecting data and processing data categories. 

\subsubsection{Collecting Data}
Data is generated and collected from different sources and different types.  This causes data to be either structured or unstructured, complex and to be data in doubt. The two Vs which make the collected data category are variety and veracity. 

\begin{itemize}
    \item Variety
    
    Big data can have different types and formats such as text, pdf, audio, tweets and images. This makes the data too complex to manage and process. The COVID-19 Open Public Media dataset is collection of 200 000 text articles with CSV files, which consist of different data types and formats.  The metadata of articles include contexts, urls, dates and numbers. Such variety in this dataset represents Big Data\cite{Johnson et al}
    \item Veracity
    
    Big data veracity is the biases, noisy and the abnormality in the data \cite{Owais}.  It involves the assessment of trustworthiness, authenticity, origin and reputation of the data. Since data is collected from different sources its quality may be compromised hence it must be careful verified before put into use. Similarly, the COVID-19 Open Public Media dataset can also viewed as a data in doubt. The data’s integrity, consistency and completeness must be analysed. The uncertainty and poor quality in the data would compromise the insights which must be leveraged from the data. 
\end{itemize}

\subsubsection{Processing Data}
Velocity and volume are the two characteristics that define processing of big data.  The rate at which data is generated from different sources determines the size of data at a given period. 
\begin{itemize}
    \item Velocity
    
    	Velocity refers to the speed at which data is generated and moves from one device to another.  The data can be generated in real time, online and offline, in streams or batches. For instance millions thousands and millions online articles, tweets and facebook massages are uploaded and posted, it must be processed instantly. 
    	
    	
    	Conventional resources are not capable of handling and processing such high speed data flows. The COVID-19 data not a live dataset, we cannot determine the rate at which it was generated. However, it cannot be processed by traditional resources due to its size and structured.
    	
    	\item Volume
    	
    	Volume is the size of data generated by either on-line or offline transactions and stored in records, tables or as files. The data size can be defined in megabytes, gigabytes, terabytes or zettabytes.   Such available and voluminous data must be ingested, analysed and managed to extract valuable insight for decision making. \cite{Owais} asserted that the large volume of big data’s primary objective is to optimize the future results. The COVID-19 dataset is a large dataset which can be used to manage the spread of novel coronavirus disease.  Researchers can leverage the dataset to mine useful information from a collection online articles and make sense out of such random information. 
    	
    	However the challenge is on analysing and processing such large data to extract valuable information. The dataset is 2G in size and unstructured which makes it difficult to process using conventional systems \cite{Patel et al}. The dataset requires big data technologies for both storage and processing; such a google cloud, HDFS and MapReduce to ensure that it is put to valuable sure in fight the spread of COVID-19 pandemic.
\end{itemize}

\begin{thebibliography}{9}

\bibitem{Wang et al}
\textit{Wang L.L, Lo K, Yang J, Reas R and Funk k. (2020). COVID-19: The COVID-19 Open Research Dataset.}

\bibitem{Patel et al}
\textit{Patel A.B, Biria M and Nair U. (2012). Addressing Big Data Problem Using Haddop and Map Reduce.}

\bibitem{Owais}
\textit{Owais S.S and Hussein N.S. (2016). Extract Five Categories CPIVW from the 9V's Characteristics of the Big Data. IJACSA, vol 7, no. 3.}

\bibitem{Johnson et al}
\textit{Johnsosn A, Havinash P.H, Paul V. and Sankaranayanan P.N. (2015). Big Data Processing Using Hadoop MapReduce Programming Model. IJCSIT, vol 6(1),127-132.}

\bibitem{Ishwarappa}
\textit{Ishwarappa and Anuradha J. (2015). A Brief Introduction to Big Data 5vs Characteristics and Hadoop Technology. Procedia Computer Science 48, 319-324.}

\bibitem{Grolinger}
\textit{Grolinger K, Hayes M, L'Heureux A, Higashino W.A and Allison D.S. (2014). Challenges for MapReduce in Big Data.}
\end{thebibliography}

\end{document}
