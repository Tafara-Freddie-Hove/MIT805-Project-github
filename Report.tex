\documentclass[12pt,letterpaper, twoside]{article}
\usepackage[utf8]{inputenc}
\usepackage[english]{babel}
\usepackage{dirtytalk}

\title{COVID-19 Open Data : Map-Reduction and Visualisation}
\author{Tafara Freddie Hove \\
        \small u18278150 \\
}
\date{}

% K
\providecommand{\keywords}[1]
{
  \small	
  \textbf{\textit{Keywords---}} #1 
}

\begin{document}
\maketitle

\begin{abstract}
COVID-19 data analytics is very important to every country to prevent loss of life. Many countries have been collecting COVID-19 data  on daily basis since the beginning of the pandemic. COVID-19 data has many and various attributes such as country name, region/states, confirmed cases, deaths and temperature. The data is being collected and archived in large volumes and different forms. Such data is rich in information which can be used to monitor and manage the spread of the pandemic. However, processing and Analysing dasatets of such huge magnitude is a big challenge. The aim of of the study is the analyze the coronavirus attributes the data attributes to observe the wide spread of COVID-19. The objectives are achieved by data processing and visualization. 

Big data technology Apache Spark was to handle the problem of big data since it uses distributed computing. Mapreduce is a framework often used for processing giant quality of knowledge Hadoop or Apache Spark cluster. A mapreducing model was used in this project to analyze COVID-19 data to give insight about the patterns and trends of active, confirmed, recovered and deceased cases in the world and South Africa in particular. The objective of this project is process, analyze and visualize the trends and patterns of the COVID-19 pandemic, and gain insights on how to curb the spread of the pandemic.

\end{abstract}\hspace{10pt}
\keywords{COVID-19, Apache Spark, Mapreduce, Visualisation}

\section{Introduction}
COVID-19 has directly impacted on human life in recent months globally. The social and economic activities of many countries negatively affected by the pandemic. Human lives were lost and livelihoods were ruined during the crises.  Due to this COVID-19 data analytics is a very significant and crucial domain for study in order to understand the devastating effects of the pandemic and make insightful decisions. Hence across the globe, there is a massive interest in COVID-19 data analysis find out where and who is mostly affected, and what measures can be implemented to 'flatten the curve' and mitigate the socio-economic imapct of the pandemic. 

The aim of this study is to perform analysis on COVID-19 dataset using Apache Spark Mapreduce. The data is a huge data which consist of many rows and columns that makes it difficult to process on using a local machine. Hence using Hadoop cluster makes it easy to process the data and get instant results without any out of memory error problems. The COVID-19 Open dataset is a Google Cloud BigQuery dataset that requires massive memory and disk space to process. The data is begin created and updated daily on the google platform. Hence such massive data is difficult to work with it requires parallel processing. 

Mapreduce is very efficient for batch processing and it can be implemented on Apache Spark which is very efficient for iterative in-memory computing and this combination simplifies supports distributed and parallel applications that handle giant quantity information. Mapreduce is a software framework  that can process massive datasets in a distributed environment over several devices \cite{}. The main idea behind Mapreduce mapping the data in a collection of (key, vaue) pairs and reducing all the pairs using the same key. Map fuction can do grouping, filtering and sorting of data. on the other hand, reduce function aggregates and summarizes the data results generated by the map function. 

In the project mapreducing and data visualisation, the COVID-19 data analytics was done by calculation average cases viz active cases of different countries. Then visualations was also conducted to assess the trends and patterns showing the spread on the pandemic. 

\section{Dataset}
 As discussed in PART 1 of the;  COVID-19 Open Data is a huge dataset that consists of country-level datasets of daily time-series data about COVID-19 worldwide \cite{covid-19}. The repository contains datasets of more than 50 countries around the world for the timespan February 2020 to date. The datasets reveal the impact of the virus and how different countries are responding to the pandemic.  It contains the latest available public data on COVID-19 including a daily situation update, the epidemiological curve and the global geographical distribution \cite{covid-19}. The COVID-19 Open Data is available at https://github.com/GoogleCloudPlatform/covid-19-open-data. This is dataset is made up of live data filesin the Google Cloud directory.

The COVID-19 Open Data is collected from multiple sources such as Wikidata, DataCommons, WorldBank, University of Oxford and Google. The data is sourced from different countries in collaboration with the World Health Organisation 's Epidemic Intelligence on daily basis. The data reports are based on the number of COVID-19  confirmed, recovered, tested and deaths cases, from health authorities worldwide. Hence the dataset is a resource of multiple types of data outcomes, static co-variate data, dynamic co-variate data and dynamic intervention state data\cite{covid-19}.The data is stored in separate CSV and JSON files which are published in Google Cloud Storage. The collection of such huge quantities of files constitutes 1.01GB of data.  The merged dataset consists of more than 5500000 observations and 45 attributes. Hence it is a massive dataset that is stored on BiQuery Puplic datset program.
We give a brief overview of ame of the attributes to be analzed using mapreduce:
\begin{enumerate}
    \item cumulative confirmed
    \item cumulative recovered
    \item cumulative deceased
    \item new ----
\end{enumerate}

\section{Methodology}
Apache Spark open source platform was used for processing data on the Google Cloud Platform. Spark is fast, scalable, fault tolerant and distributed. Hence it is referred to as the general engine for processing large scale data  \cite{Chouksey and Chauhan, 2017}. Hence Spark was implemented for mapreducing. Mapreduce framework efficiently distributes the job over a number of commodity hardware and calculate the result in parallel. Spark can efficiently analyzes large data sets since it relies on in-memory storage for computing. Its main core data structure is RDD which is a dataset distributed across the RAM  of each computer in a cluster. The calculation is in Spark are executed through pipelining using transformation (map, reduceByKey) and action (take, reduce, collect) operations. To ensure effective COVID-19 data analytics using Spark Mapreduce the number of operations were implemented.

\subsection{Cloud Dataproc Cluster Setup}
Firstly, the project was created on Google Cloud platform console. The cloud storage bucket to be used in the dataproc cluster was created to store data files from BigQuery . The dataproc cluster with component gateway was officially created. The notebook was also created by making use the spark BigQuery storage connector to ensure that the data will be easily accessed fro the BiqQuery Storage public dataset. 

The Dataproc hadoop cluster with Apache Spark was setup on the three nodes with one master (namenode) and two worker nodes(datanodes). The cluster was created in a difined region and zone. Each node had 2V CPU3.75GB n1-standard core, and 32GB disk size storage capacity. Then the latest version of hadoop 2.9 and spark 2.4 was used to setup the cluster. In a dataproc hadoop cluster, the jupyter notebook for python was was also accessed for the processing of big data.  In the dataproc cluster notebook the spark-bigquery-connector and BigQuery Storage API were used to  load data into the Spark Cluster. Then the Spark data from can be created by reading in data from the public BigQuery dataset. 

\subsection{ Exploratory Data Analysis}
Data exploration analysis is the very first and critical step in data analysis process. It involves the exploring of the data to identity the data patterns, trends,, variable characteristics and other interesting points. it helps to understand the key concepts and issues of the data so as to get a better informed representation of the entire dataset. Through data exploration on notebooks the dimensions of the dataset and the variable types were identified. Hence the data set was cut into a manageable size, focusing on analyzing the most relevant attributes.  

\subsection{Data preparation}
Data preparation involves cleaning, transforming and rearranging of data. The EDA revealed that the dataset consists of 5500000 rows and 45 columns and some of the data observations are incomplete. The dataset had missing data and duplicate data. The dublicate columns were removed, and the filling in of missing data was also done. The selection of the describing important  attributes was done. Hence the data was rearranged in various ways to ensure  manipulation of datasets. 


\subsection{Spark MapReduce}
Spark is an efficient and fast engine used to process and analyze large-scale data. The Spark can implement Mapreduce data flow pattern on the jupyter notebook. The computations of the pre-processed  COVID-19 dataset were distributed over different worknodes on the cluster by the spark's RDD. Then transformation and action methods were implemented to analyze the data using the selected data features.

define transformations, actioons, give exaples

which attributes were slected an why????

\section{Results}
\section{Visualisation}
\end{document}
